\section{Language Evolution}

The previous sections have shown how to build a language extension for mbeddr in
\ac{MPS}, define a debugger for this extension and use \ic{DeTeL} to test its
debugging behavior.
This section demonstrates how \ic{DeTeL} is used to locate invalid definitions
in debugger extensions after evolving the language.

\subsection{Evolving MUnit}

In this section we modify the MUnit generator to reduce the amount of C
code that it generates. Currently, the generator reduces an
\ic{ExecuteTestExpression} to a \ic{FunctionCall} that calls a helper
\ic{Function}, which then calls the reduced \ic{Testcase}s (see \lst{lst:generatedUT}).
We modify this generator, so \ic{ExecuteTestExpression} is reduced to
\ic{FunctionCall}s, which directly call the reduced \ic{Testcase}. In case of
referencing more than one \ic{Testcase}, the \ic{FunctionCall}s are concatinated
via \ic{PlusExpression}s to return the overall number of failed
\ic{Testcase}s. The listing below shows for the \emph{main} \ic{Function}
from \lst{lst:generatedUT} how our generator change affects the generated code. 

\vspace{-2mm}
\noindent 
\hspace{1.2mm}
\begin{minipage}[t]{120pt} 
\begin{lstlisting}[language=reducedMbeddr,numbers=left]
int32 main(int32 argc,
		string[] argv) {
   return $\colorbox{g1}{test[}$$\colorbox{g6}{forTest}$$\colorbox{g1}{]}$;
}
\end{lstlisting}
\end{minipage} 
\rule[-10ex]{0.1ex}{4.0em}
\hspace{0.75mm}
\begin{minipage}[t]{125pt} 
\begin{lstlisting}[language=reducedMbeddr,numbers=left]
int32_t main(int32_t argc,
		char *(argv[])) {
   return $\colorbox{g6}{test\_forTest()}$;
}  
\end{lstlisting}
\end{minipage} 
\vspace{-4mm}
\begin{lstlisting}[caption={Parts of the example program
from \lst{lst:generatedUT} using \emph{MUnit} on the left and the C code
generated form it with our modified generator},
language=mbeddr,label=lst:newGeneratedUT]
\end{lstlisting}



Because of our generator modification, \ic{ExecuteTestEx- pression} is not
reduced anymore to a \ic{Function}. However, we have not updated the
debugger extension, therefore, the call stack construction for all test cases
will fail and this way the tests will fail as well. Altough those debugger tests
fail, they are still valid, because they are written on the abstraction level of
the languages, not the generator. The next section shows how we update the
debugger extension to solve the call stack construction problem.

\begin{figure}[h]
	\vspace{-2mm}
	\centering
    \includegraphics[width=8.5cm]{./figures/failingDebuggerTests.png} 
    \vspace{-3mm}
	\caption{Failing \ic{DebuggerTestcase}s after modifying the generator}
	\label{fig:TestExecution2}
	\vspace{-2mm}
\end{figure}

\subsection{Updating the Debugger Extension}

Because \ic{ExecuteTestExpression} is generated to an \ic{Expression} containing
calls to the referenced \ic{Testcase}s, it is not generated to a \ic{Function} anymore
and does therefore not contribute a \ic{StackFrame} anymore. To solve the
problem, \ic{ExecuteTestExpression} simply does not implement
\ic{StackFrameContributor} anymore. Other aspects such as stepping, breakpoints
and watches are not affected by the generator modification and hence do not need
to be changed. After removing the interface implementation, all of our debugger
tests pass again.