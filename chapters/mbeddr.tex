\section{mbeddr}


The mbeddr open source project focused on supporting embedded software development. It introduces a set of of modular domain-specific extensions to C and also supports other languages for addressing common problems in software development, for example, writing documentation with close integration to code or capturing requirements. mbeddr is build using JetBrains \ac{MPS} language workbench. \ac{MPS} supports the definition, composition and use of general purpose or domain specific languages. To archive this \ac{MPS} uses a projectional editor, this means, even if the notation might look textual it is not represented as a sequence of characters, which are transformed into a \ac{AST} by parsing. In contrast user actions manipulate the \ac{AST} directly. The \ac{AST} is then rendered to the user according to editor specifications/projection rules. These rules are not limited to textual notations, for example can tabular or mathematical notations can be used if appropriated. Since no parsing ambiguities can occur a wide range of languages extension can be used.

\todo{das komische mbeddr diagram hier rein}
\todo{ ref auf  Markus Voelter. Language and IDE Development, Modularization and Composition with MPS. In GTTSE 2011, LNCS. Springer, 2011.}

\subsection{Languages}
\label{languageImplementation}
mbeddr includes a extensible C99 implementation. In addition to plain C mbeddr also include a set of predefined extension on top of C. These extension include test cases, state machines, components and physical units. In \ac{MPS} languages are separated into modular aspects. The major aspects of a language are:  

\parhead{Structure:} Definition of the \ac{AST} of the language.

\parhead{Editor:} Projection rules how the \ac{AST} is presented to the user and how the user interacts with the program.

\parhead{Typesystem/Contrains:} Static semantics of the language.
 
\parhead{Generator:} Dynamic semantics of the language, transforms the model into executable code.


C + Extensions 
	    -> Lang ex +  Generator + Debugger
		+ Sample von Extension (focus on Generator)
		
\subsection{Foreach/Unit Test Example}
