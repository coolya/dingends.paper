\section{Related Work}

Wu et al. describe a unit testing framework for 
\acp{DSL}~\cite{DBLP:conf/dsl/WuGM09} with focus on
testing the semantics of the language.  However, from our perspective, it is
necessary to have testing \acp{DSL} for all aspects of the language definition, \eg editor
(concrete syntax), type system, scoping, transformation rules, and finally the
debugger.\footnote{Specific language workbenches might require testing of additional aspects}
mbeddr contains tests for the editor, type system, scoping and transformation
rules, our work contributes the language for testing the debugger aspect.

The \ac{LLVM} project~\cite{LLDB} comes with a C debugger named \ac{LLDB}.
Test cases for this debugger are written in Python
and the unit test framework of Python. While those tests verify the 
command line interface and the scripting \ac{API} of the debugger, they also
test other functionality, such as using the help menu or changing the debugger settings.
Further, some of the \ac{LLDB} tests verify the debugging behavior on different
platforms, such as Darwin or Linux. In contrast, we only concentrate on
testing the debugging behavior, but also support writing tests for
specific platforms. Our approach for testing the debugging behavior is
derived from the \ac{LLDB} project: write a program in the source-language
(mbeddr), compile it to an executable and debug it through test cases, which verify the
debugging behavior.

The \ac{GDB} debugger takes a similar approach as
the \ac{LLDB}: tests cover different aspects of the debugger
functionality and are written in a scripting language~\cite{gdb}.
Contrarily, to our approach of testing debugging 
for one extensible language, the \ac{GDB} project tests
debugging behavior for all of its supported languages, such as C, C++, Java, Ada
etc. Further, those tests run on different platforms and target
configurations. Our work supports writing tests against different platforms, but
does not allow users to change the target configuration via the \ac{DSL}.
