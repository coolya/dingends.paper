\section{Related Work}

Wu et al. describe a unit testing framework for
\acp{DSL}~\cite{DBLP:conf/dsl/WuGM09}. 
While they concentrate on testing the semantics of the language, we believe testing
\acp{DSL} should not only cover the semantics, but also other language
aspects: editor (concrete syntax), type system, scoping, transformation rules,
and finally the debugger.\footnote{Specific language workbenches might require
testing of additional aspects} Those aspects also
form part of the language implementation and should therefore be tested as well.

The \ac{LLVM} project~\cite{LLDB} comes with a C debugger named \ac{LLDB}.
Tests for this debugger are written in Python and with the 
unittest framework that comes with Python.
While those test verify the debugger's command
line interface and the scripting API, they also test other functionality, such
as using the help menu or changing the debugger settings.
Further, some of those tests
verify the debugging behavior on different platforms, such as Darwin, Linux or
FreeBSD.
In contrast to the \ac{LLDB} project, we only concentrate on testing
the debugging behavior. The approach for doing so is derived from the \ac{LLDB} 
project:
write a program in the source-language (mbeddr), compile it to an executable and
debug it through test cases, which verify the debugging behavior. While the 
\ac{LLDB} project also tests other aspects with Python scripts, we could invent
further \acp{DSL} for testing those aspects as well.

The \ac{GDB} provided by the GNU project takes a similar
appraoch as the \ac{LLDB}: debugger tests are cover different aspects of the
debugger's functionality and are written in a scripting language~\cite{gdb}.
While we concentrate on testing the debugging behavior for one extensible
language, the \ac{GDB} project tests debugging behavior for all of its supported
languages, such as C, C++, Java, Ada etc. Further, those tests run on different
platforms and different target configurations. In contrast, we only also support
writing tests against different platforms, but do not allow users to change the
target configuration via the \ac{DSL}.
