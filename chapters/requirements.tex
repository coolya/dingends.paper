\section{Requirements}
\label{DesignDecisions}

The debugger testing \ac{DSL} must allow us to verify at least
four aspects: call stack, program state, breakpoints and stepping.
To cover these requirements in \emph{DeTeL} we delineate in this section
requirements and their implementation strategy. While we
consider some of those requirements as required (\textbf{R}), others are either
context (\textbf{CS}) or mbeddr specific (\textbf{MS}).

\subsection{Required}

\textbf{\label{R1}R1 Debug state validation:} 

Changes in generators can modify names of generated procedures or
variables and this way, \eg invalidate program state lifting in the the
debugger.
For being able to identify those problems, we need a  possibility to validate
the call stack, and for each of its frames the program state and the location where
execution is suspended. For the call stack,
a specification of expected stack frames with their respective names is
required. In terms of program state, we need to verify the names of watches 
and their respective values, which can either be simple
or complex. Further, a location specifies where program execution is
expected to suspend and tests can be written for a specific platform. 

\textbf{\label{R2}R2 Debug control:} Similarly as in \hyperref[R1]{R1},
generator changes also affect the stepping behavior. 
Consider changing the \ic{FunctionCall} generator to inline called
functions instead of calling them. This change would require modifications in
the implementation of \emph{step into} as well. For being able to identify
those problems, we need the ability to execute stepping commands (in, over
and out) and specify locations where to break.

\textbf{\label{R3}R3 Language integration:} The
\ac{DSL} must integrate with language extensions.
This integration is required for specifying in programs under test
locations where to break (see \hyperref[R2]{R2}) and for validating where
program execution is suspended (see \hyperref[R1]{R1}).

\subsection{Context Specific}

\textbf{\label{CS1}CS1 Reusability:} For writing debugger tests in
an efficient way, we expect from \emph{DeTeL} the ability to provide reuse: (1)
test data, (2) validation rules and (3) the structure of tests. The first covers
the ability to have one mbeddr program as test data for multiple test cases.
The second refers to single definition and multiple usage of validation rules
among different test cases. Finally, the third refers to extending test cases
and having the possibility to specialize them.

\textbf{\label{CS2}CS2 Extensibility:} 

Languages should provide support for contributing new
validation rules thus achieving extensibility. Those new rules can be used for
testing further debugger functionality not covered by \emph{DeTeL} (\eg mbeddr's
upcoming support for multi-level debugging~\cite{MultiLevelDebugging:WSRE:breakedForInlining}) 
or for writing tests more efficiently.


\textbf{\label{CS3}CS3 Automated test execution:} For fast
feedback about newly introduced debugger bugs, 
we require the ability to integrate our 
tests into an automatic execution environment (\eg an \ic{IDE} or a build
server).

\subsection{Mbeddr Specific}

\textbf{\label{MS1}MS1 Exchangeable debugger backends:}

mbeddr targets the embedded domain where platform vendors require different
compilers and debuggers. Hence, we require the ability to run our tests against
different debugger backends and on different platforms. 
