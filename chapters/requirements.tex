\section{Requirements on the Testing Language}

Our testing \ac{DSL} targets debuggers for imperative extensible languages.
Those debuggers usually provide at least the following functionality:
show the call stack and program state in relation to the languages used, 
suspend the debugger on breakpoints and offer the possibility to 
step through the code. Our testing language should allow us to test exactly those
four aspects. We have therefore defined the following functional and
non-functional requirements on the \ac{DSL}, which we will later implement in
our language. While some of those requirements are optional, we consider most of
them as mandatory.

\subsection{Functional Requirements}

The following list discusses the functional requirements we expect from a
testing language for debuggers. While \hyperref[FR3]{FR3} is considered
optional and \hyperref[FR3]{FR3} is mandatory in the embedded software domain,
we expect \hyperref[FR1]{FR1} and \hyperref[FR2]{FR2} to be essential.
 
\noindent \textbf{\label{FR1}FR1 Debug state validation:} In our test cases we
must be able to validate the call stack, and for each of its frames 
the program state and the location where suspended. 

The stack itself should be validated by specifying the expected stack frames and
their respective names. In terms of program state, we need to verify the names 
of watchables and their respective values, with can either be simple or complex.
Finally, location where suspended is a reference to a location inside the program under tests. 

\noindent \textbf{\label{FR2}FR2 Debugger control:} In order to test stepping
commands and breakpoints, we need the ability to specify stepping commands
(in, over and out) and locations where to break.

In a debugger test cases,
first, the location where to break is specified. Next, after suspending on this
location, an optional list of stepping commands is executed. Finally, after
suspending, specified validation rules are evaluated.

\noindent \textbf{\label{FR2}FR2 Automated test execution:} Since we can write
many different debugger tests, we require the ability to integrate our 
tests into an automatic execution environment. This environment can be, \eg
an \ac{IDE} or a command-line tool. Latter is preferred, since it can be
executed on the local machine or on a build server.

\noindent \textbf{\label{FR3}FR3 Exchangeability of debugger backends:}
mbeddr targets the embedded domain and C. In this domain, target platforms and
vendors require different compilers and debuggers. 
Hence, we require the ability to run our tests against
different debugger backends and on different platforms. To achieve this,
we expect from the \ac{DSL} the ability to specify the debugger backend
and the platform against/on which tests should executed.

\subsection{Non-Functional Requirements}

In addition to the functional requirements discussed in the previous section, we
have also defined non-functional requirements: \hyperref[NFR1]{NFR1} to
\hyperref[NFR3]{NFR3}. While \hyperref[NFR2]{NFR2} can be considered optional,
we consider \hyperref[NFR1]{NFR1} and \hyperref[NFR3]{NFR3} as mandatory for
testing debugger implementations efficiently.

\noindent \textbf{\label{NFR1}NFR1 Reusability:} For writing debugger tests in
an efficient way, we expect from the language the ability to reuse (1) test data, (2) 
validation rules and (3) the structure of tests. First covers the ability to
have one mbeddr program to debug and write different debugger tests against
it. Second refers to reusing a set of validation rules in different test cases.

Consider different test cases testing stepping behavior inside the \ic{main} 
function of a C program.
As long as we stay inside the function, the call stack (stack frames, not the
location where suspended) would be the same for each test case. Hence, we could
write the C program and the call stack validation rule once, and refer to it
from each of our test cases. Further, we could even think about writing
an abstract test case, which is specialized by our actual test cases.


\noindent \textbf{\label{NFR2}NFR2 Extensibility:} As discussed in
\hyperref[FR1]{FR1}, 
we expect from the testing language the ability to validate the program state, location where
suspended and the call stack. In addition to this, we
require the ability to extend the testing language for contributing our own
validation rules. Those new rules can be used for testing debugger
functionality not covered by the language or for writing tests more efficiently.

\noindent \textbf{\label{NFR3}NFR3 Source language integration:} Finally,
\ac{DSL} should integrate with programs written in the source language to debug.
This integration is required for specifying locations to break and for
validating where the debugger is suspended.
