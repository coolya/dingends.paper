\section{Requirements on the Testing Language}

Our testing \ac{DSL} targets debuggers for imperative extensible languages.
Those debuggers usually provide at least the following functionality:
show the call stack and program state in relation to the languages used, 
suspend the debugger on breakpoints and offering step into, over and out to step
through the code. Our testing language should allow us to test exactly those
aspects. We have therefore defined the following functional and non-functional
requirements on the \ac{DSL}. While some of them are optional, we consider most
of them as mandatory.

\subsection{Functional Requirements}

The following list discusses the functional requirements we expect from a
testing language for debuggers. While \hyperref[FR2]{FR2} is considered
optional and \hyperref[FR3]{FR3} is required in the embeded domain, we expect
\hyperref[FR1]{FR1} to be essential.

\noindent \textbf{\label{FR1}FR1 Validating debugging behavior:} Stepping,
Program State, Call Stack, 	Breakpoints

\noindent \textbf{\label{FR2}FR2 Automated test execution:} As we can write many
different debugger tests, the \ac{DSL} or code generated by it should be
integratable in an automatic execution environemt. This environment can either
be an \ac{IDE} or a command-line program, which can be executed on the local
machine or on a build server.


\noindent \textbf{\label{FR3}FR3 Exchangeability of debugger backends:}
mbeddr targets the embedded domain and C. In the embedded domain, target platforms and
vendors require different compilers and debuggers. 
Hence, we require the ability to run our tests against
different debugger backends. From the testing language we expect the ability to
specify the debugger backend against which tests should executed. Further, we
must be able to run one test case against different debugger backends.

\subsection{Non-Functional Requirements}

In addition to the functional requirements discussed in the previous section, we
have also defined sone non-functional requirements: \hyperref[NFR1]{NFR1} to
\hyperref[NFR3]{NFR3}. While \hyperref[NFR2]{NFR2} can be considered optional,
we consider \hyperref[NFR1]{NFR1} and \hyperref[NFR3]{NFR3} as mandatory 
for efficiently testing debugger implementations.

\noindent \textbf{\label{NFR1}NFR1 Reusability:} For writing debugger tests in
an efficient way, we expect from the language the ability to reuse (1) test data, (2) 
validation rules and (3) the structure of tests. First covers the ability to
have one mbeddr program to debug and write different debugger tests against
it. Second refers to reusing specified validation rules in different tests.
Consider different tests testing stepping behavior inside the same C function.
The call stack (stack frames, not the location where suspended) would be the
same in each test case. Finally, third involves the ability to extend other test
cases and the possibility to overwrite parts of it. 
 

\noindent \textbf{\label{NFR2}NFR2 Extensibility:} As discussed in
\hyperref[FR1]{FR1}, 
we expect from the testing language the ability to validate the program state, location where
suspended and the call stack. In addition to this validation functionality, we
require the ability to extend the testing language for contributing our own
validation rules. Those new rules can be used for testing debugger
functionality not covered by the language or for writing tests more efficiently.

\noindent \textbf{\label{NFR3}NFR3 source langauge integration:} We exepect from
the testing \ac{DSL} to integrate with programs written in the source language to debug.
This integration is required for specifying locations to break and for
validating where the debugger is suspended.
