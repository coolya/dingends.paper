\section{Introduction}

Software development faces the challenge that \acp{GPL}
do not provide the appropriate abstractions for domain-specific problems. 
Traditionally there are two  
approaches to overcome this issue. One is to use frameworks 
that provide domain-specific abstractions expressed with a \ac{GPL}.
This approach has very limited support for static semantics, \eg 
no support for modifying constraints or type system.
% first appraoch is about internal DSLs?
Second approach is to use external \acp{DSL} for 
expressing solutions to domain problems. This approach 
has some other drawbacks: 
these \acp{DSL} are not inherently extensible.
% what was here: monolithic, all in advanced, central, foo bar
Extensible languages solve these problems. Instead of having a single 
monolithic \ac{DSL}, extensible languages enable modular and 
incremental extensions of a host language with domain specific 
abstractions~\cite{Voelter2011}.

To make debugging extensible languages useful to the language user, it is not
enough to debug programs after extensions have been translated back to the host 
language (using an existing debugger for the base language).  
A debugger for an extensible language must be extensible as well, to support
debugging of modular language extensions at the same abstraction level
(extension-level).
Minimally, this means that users can step through the constructs 
provided by the extension and see watch expressions related to the extensions.

% fix this part (we might have a fitting sentence somewhere)
Because language extensions can be based on other extensions and languages
evolve over time, it is essential to constantly test if debugger
behavior matches the expected behavior. To ensure debugging behavior with
testing, a \ac{GPL} can used, however this raises the same issues
discussed above. 
% ToDo: double check it!
We therefore propose in this paper a \ac{DSL} 
for testing debuggers. 
